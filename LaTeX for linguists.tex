\documentclass{article}

\usepackage[utf8]{inputenc}
\usepackage{tipa}
\usepackage{tipx}
\usepackage{vowel}
\usepackage{phonrule}
\usepackage{qtree}

\usepackage{CJKutf8}
\usepackage{xpinyin}
\usepackage{ruby}
\renewcommand{\rubysep}{\defaultrubysep}

\title{LaTeX for linguists}
\author{Paulina Pieper}
\date{April 2021}

\begin{document}

\section{}
%http://www.l.u-tokyo.ac.jp/~fkr/tipa/tipaman.pdf

\subsection{\textipa{'INglIS}}
\textipa{/maI 'neIm Iz paw'lIna/
/aIm 'twEnti 'fO: 'jI@z @ld/ /aI 'lIv In 'gdansk/ /aI 'gr\ae djUeItd In 'A:t 'hIst@ri. naU aIm 'st2diN 'INglIS @n 'n\ae \textteshlig r@l 'l\ae NgwI\textdyoghlig  'pr@UsEsIN/}
\bigskip

\subsection{\textipa{'pOl\|[sk\super ji}}
\textipa{/'mam \|[na 'im\super ji\~E paw'l\super jina/
/mam \|[dva'\t{d\textctyogh}EC\t{tC}a '\t{t\:s}\|[tEr1 'la\|[ta/
/'ukO\textltailn\t{t\:s}1wam 'x\super ji\|[s\|[tOr\super ji\~E '\:s\|[tuki
'\|[tEra\|[z \|[s\|[tu'\|[d\super jiuj\~E 'filOlOg\super ji\~E 'a\|[ng\super jiEl\|[sk\~O zE \|[spE'\t{\|[t\|[s}jal\|[nOC\~O f p\:sE'\|[tvaza\|[n\super ju 'jE\|[z1ka \|[na'\|[tural\|[nEgO/}

\bigskip

\section{Standard Chinese vowels}

\begin{vowel}
\putcvowel[l]{i}{1}
\putcvowel[r]{y}{1}
\putcvowel{\textlowering{e}}{2}
\putcvowel{\textbari}{9}
\putcvowel{a}{4}
\putcvowel[l]{\textramshorns}{7}
\putcvowel[r]{o}{7}
\putcvowel{u}{8}
\putcvowel{\textrhookschwa}{11}
\end{vowel}


\section{}

%https://sunsite.icm.edu.pl/pub/CTAN/macros/latex/contrib/phonrule/phonrule-doc.pdf
\phonc{/l/}{[\texttoptiebar{\textsubring{l}l}]}{
\phonfeat[l]{
+consonant \\
-voice}
} \phold \\
\\
\\
\phonr{/l/}{[\textsubbridge{\textltilde}]}{\texttheta}

\newpage
\section{}
%https://www.ling.upenn.edu/advice/latex/qtree/qtreenotes.pdf
Gomez may have been keeping quiet and minding his own business.

\Tree [.S \qroof{ Gomez}.NP [.VP [.MOD may ] [.VP [.PERF have ] [.VP [.PROG been ] [.VP [.VP [.V keeping ] [.AP [.A quiet ] ] ] [ and ] [.VP [.V minding ] \qroof{his own business}.NP ] ] ] ] ] ]


\section{}

\begin{CJK*}{UTF8}{min}
\subsection{日本語でセクション}
これは日本語でセクションです。何をここで書けばいいのか分からないでも、\LaTeXに日本語でやってみたい。
\\
\\
%https://tex.stackexchange.com/questions/95729/typesetting-furigana-above-and-below-original-text
これは\ruby{日}{に}\ruby{本}{ほん}\ruby{語}{ご}でセクションです。\ruby{何}{なに}をここで\ruby{書}{か}けばいいのか\ruby{分}{わ}からないでも、\ruby{\LaTeX}{ラテック}に\ruby{日}{に}\ruby{本}{ほん}\ruby{語}{ご}でやってみたい。
\end{CJK*}

\begin{CJK*}{UTF8}{gbsn}

\subsection{汉语的段}
这是汉语的段。我不知道什么写这里。但是我想看看汉语的\LaTeX。
\\
\\
%https://sunsite.icm.edu.pl/pub/CTAN/macros/latex/contrib/xpinyin/xpinyin.pdf
\begin{pinyinscope}
这是汉语的段。我不知道什么写这里。但是我想看看汉语的\LaTeX。
\end{pinyinscope}

\end{CJK*}

\end{document}
